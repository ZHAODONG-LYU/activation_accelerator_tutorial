\documentclass[conference]{IEEEtran}
\IEEEoverridecommandlockouts

% --- Packages ---
\usepackage[utf8]{inputenc}
\usepackage{graphicx}
\usepackage{amsmath}
\usepackage{amssymb}
\usepackage{booktabs}
\usepackage{cite}
\usepackage{url}
\usepackage{array}
\usepackage{xcolor}
\usepackage{dirtree}

% --- Document ---
\begin{document}

\title{FPGA Innovation Design Contest AMD Track -- FPT International Conference Design Report*\\ 
\large *Change your paper title}

\author{\IEEEauthorblockN{Anonymous Authors}}

\maketitle

\begin{abstract}
This paper presents a project developed for the 2025 National Undergraduate Embedded Chip and System Design Competition -- FPGA Innovation Design Contest, AMD Track (Advanced Self-Selected Topic). The abstract should summarize the background, main contributions, technical approach, experimental results, and conclusions within 300--500 words.
\end{abstract}

\begin{IEEEkeywords}
FPGA, hardware acceleration, embedded systems, parallel computing, AMD
\end{IEEEkeywords}

\section{Introduction and Research Background}

\subsection{Research Background}
This section introduces the technical background and application domain of the project. It analyzes existing computing paradigms such as CPUs, FPGAs, GPUs, and ASICs, comparing their advantages and disadvantages. The section should identify the limitations of current technologies and the potential for innovation.

\subsection{Main Contributions}
The main contributions of this project are summarized as follows:
\begin{enumerate}
    \item \textbf{Contribution 1:} [Detailed description]
    \item \textbf{Contribution 2:} [Detailed description]
    \item \textbf{Contribution 3:} [Detailed description]
\end{enumerate}

\subsection{Report Structure}
This paper is organized as follows: Section II presents the design methodology and key technologies. Section III discusses experimental design and performance evaluation. Section IV provides comparative analysis. Section V outlines challenges, lessons learned, and future directions.

\section{Design Methodology and Key Techniques}

\subsection{Overall Design Concept}
This section describes the overall design philosophy and methodology of the system.

\subsection{Key Algorithm Design}
The design and optimization of the core algorithms are elaborated here.

\subsubsection{Theoretical Foundations}
This part explains the mathematical principles underlying the proposed algorithms.

\subsubsection{Parallel Design}
Multi-level parallelization strategies are discussed in detail.

\subsubsection{Pipeline Design and Optimization}
This section describes the pipelined architecture and optimization techniques.

\subsubsection{Data Precision Analysis and Optimization}
An analysis of the impact of numerical precision on system performance and accuracy.

\subsection{High-Performance Memory Design}

\subsubsection{Memory Access Optimization}
Optimization techniques for memory access patterns are presented.

\subsubsection{Bandwidth Optimization}
This section discusses techniques to enhance memory bandwidth utilization.

\subsection{Hardware-Software Interface Design}

\subsubsection{High-Speed Communication Interfaces}
Design considerations for high-speed data transmission interfaces are detailed here.

\section{Experimental Design and Performance Evaluation}

\subsection{Experimental Environment}

\subsubsection{Software Environment}
\begin{itemize}
    \item \textbf{Operating System:} [Version information]
    \item \textbf{Development Tools:} [Vivado/Vitis version]
    \item \textbf{Programming Environment:} [Languages and libraries]
\end{itemize}

\subsubsection{Benchmark Comparison}
This section describes the baseline implementations used for performance comparison.

\subsection{Functional Verification}

\subsubsection{Unit Testing}
Detailed testing of each module is provided.

\subsubsection{Integration Testing}
System integration testing results are presented.

\subsection{Performance Testing}

\subsubsection{Latency Testing}
Detailed latency measurement results are presented.

\subsubsection{Accuracy Verification}
Verification of algorithmic accuracy is described.

\subsection{Resource Utilization Analysis}

\subsubsection{Logic Resource Usage}
\begin{table}[h]
\centering
\caption{FPGA Resource Utilization}
\begin{tabular}{lcccc}
\toprule
Resource & Used & Available & Utilization & Evaluation \\
\midrule
LUT & [num] & [total] & [\%] & [evaluation] \\
FF & [num] & [total] & [\%] & [evaluation] \\
BRAM & [num] & [total] & [\%] & [evaluation] \\
DSP & [num] & [total] & [\%] & [evaluation] \\
\bottomrule
\end{tabular}
\end{table}

\subsubsection{Memory Resource Efficiency}
A detailed analysis of memory resource utilization efficiency.

\subsubsection{Computation Resource Efficiency}
An analysis of DSP and LUT computational efficiency.

\subsection{Performance Comparison and Analysis}

\subsubsection{Comparison with Existing Solutions}

\begin{itemize}
    \item \textbf{Performance Comparison:} Comparative performance results between this work and existing solutions.
    \item \textbf{Advantage Analysis:} A detailed analysis of the advantages of this work.
    \item \textbf{Limitation Analysis:} A discussion of the current limitations of this work.
\end{itemize}

\subsubsection{Scalability Analysis}
Analysis of scalability and adaptability of the proposed design.


\begin{thebibliography}{00}
\bibitem{b1} G. Eason, B. Noble, and I. N. Sneddon, ``On certain integrals of Lipschitz-Hankel type involving products of Bessel functions,'' Phil. Trans. Roy. Soc. London, vol. A247, pp. 529--551, April 1955.
\bibitem{b2} J. Clerk Maxwell, A Treatise on Electricity and Magnetism, 3rd ed., vol. 2. Oxford: Clarendon, 1892, pp.68--73.
\bibitem{b3} I. S. Jacobs and C. P. Bean, ``Fine particles, thin films and exchange anisotropy,'' in Magnetism, vol. III, G. T. Rado and H. Suhl, Eds. New York: Academic, 1963, pp. 271--350.
\bibitem{b4} K. Elissa, ``Title of paper if known,'' unpublished.
\bibitem{b5} R. Nicole, ``Title of paper with only first word capitalized,'' J. Name Stand. Abbrev., in press.
\bibitem{b6} Y. Yorozu, M. Hirano, K. Oka, and Y. Tagawa, ``Electron spectroscopy studies on magneto-optical media and plastic substrate interface,'' IEEE Transl. J. Magn. Japan, vol. 2, pp. 740--741, August 1987 [Digests 9th Annual Conf. Magnetics Japan, p. 301, 1982].
\bibitem{b7} M. Young, The Technical Writer's Handbook. Mill Valley, CA: University Science, 1989.
\bibitem{b8} D. P. Kingma and M. Welling, ``Auto-encoding variational Bayes,'' 2013, arXiv:1312.6114. [Online]. Available: https://arxiv.org/abs/1312.6114
\bibitem{b9} S. Liu, ``Wi-Fi Energy Detection Testbed (12MTC),'' 2023, gitHub repository. [Online]. Available: https://github.com/liustone99/Wi-Fi-Energy-Detection-Testbed-12MTC
\bibitem{b10} ``Treatment episode data set: discharges (TEDS-D): concatenated, 2006 to 2009.'' U.S. Department of Health and Human Services, Substance Abuse and Mental Health Services Administration, Office of Applied Studies, August, 2013, DOI:10.3886/ICPSR30122.v2
\bibitem{b11} K. Eves and J. Valasek, ``Adaptive control for singularly perturbed systems examples,'' Code Ocean, Aug. 2023. [Online]. Available: https://codeocean.com/capsule/4989235/tree
\end{thebibliography}



% -------------------- Appendix --------------------
\onecolumn
\appendices
\section{System Source Code Directory Tree}

\dirtree{%
.1 project\_root/.
.2 src/ : Source code directory.
.3 rtl/ : RTL code.
.4 top\_module.v : Top-level module.
.4 accelerator/ : Accelerator core.
.4 interface/ : Interface modules.
.4 utils/ : Utility modules.
.3 software/ : Software section.
.4 driver/ : Drivers.
.4 api/ : API interfaces.
.4 app/ : Applications.
.3 constraints/ : Constraint files.
.3 testbench/ : Test files.
.3 scripts/ : Build scripts.
.2 docs/ : Documentation.
.2 data/ : Test data.
.2 results/ : Experimental results.
.2 README.md : Project description.
}



\section{Key LLM Interaction Records}

\subsection{Important Interaction 1:}
\begin{itemize}
    \item \textbf{User Question:} [Simplified question]
    \item \textbf{LLM Suggestion:} [Key suggestion summary]
    \item \textbf{Implementation Effect:} [Observed improvement]
\end{itemize}


\vspace{0.5em}
\subsection{Important Interaction 2:}
\begin{itemize}
    \item \textbf{User Question:} [Simplified question]
    \item \textbf{LLM Suggestion:} [Key suggestion summary]
    \item \textbf{Implementation Effect:} [Observed improvement]
\end{itemize}


\vspace{0.5em}
\subsection{Important Interaction 3:}
\begin{itemize}
    \item \textbf{User Question:} [Simplified question]
    \item \textbf{LLM Suggestion:} [Key suggestion summary]
    \item \textbf{Implementation Effect:} [Observed improvement]
\end{itemize}

\section{Other supplement contents}

\end{document}